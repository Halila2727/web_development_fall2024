\documentclass{article}
\usepackage[utf8]{inputenc}
\usepackage{graphicx}
\usepackage{tikz}
\usepackage[colorlinks=true, urlcolor=blue, linkcolor=red]{hyperref}
\title{Final Project}
\author{CSCI395 Web Development}
\date{Due: December, 20th 2024}


\begin{document}


\maketitle


\section{Overview}
In this project, students are expected to build a website using the\\
Express/Node.js platform, with the Axios HTTP client, that integrates a chosen\\
public API from the given list: \href{https://github.com/public-api-lists/public-api-lists}{Public API Lists}.
The website should interact with the chosen API, retrieve data, and present it in a user-friendly manner.


\section{Objectives}
\begin{itemize}
  \item Develop an understanding of how to integrate public APIs into web projects.
  \item Gain practical experience using Express/Node.js for server-side programming.
  \item Enhance understanding of client-server communication using Axios.
  \item Demonstrate ability to manipulate, present, and work with data retrieved from APIs.
  \item Persist data using PostgreSQL database.
\end{itemize}


\section{Example Idea: Weather Tracker Web Application}

This web application allows users to monitor the current weather each time they log in. The application integrates with the OpenWeatherMap API to fetch real-time weather data.

\section*{Key Features}

\subsection*{1. User Data Management}
\begin{itemize}
    \item Each user’s email, password, and zip code are securely stored in a database.
    \item The zip code is used to determine the user’s location and fetch corresponding weather details.
\end{itemize}

\subsection*{2. Weather History Tracking}
\begin{itemize}
    \item A dedicated \textbf{"History"} tab records and displays all instances when users accessed the weather information.
    \item This feature provides users with a complete log of their activity.
\end{itemize}

\subsection*{3. Activity Recommendations (Bonus Feature)}
\begin{itemize}
    \item The application includes an additional tab in the navigation bar that suggests random activities based on the current temperature.
    \item Activities are sourced from two tables in the database:
    \begin{itemize}
        \item \textbf{"Warm Activities"} for higher temperatures.
        \item \textbf{"Cool Activities"} for lower temperatures.
    \end{itemize}
    \item These suggestions enhance user engagement by tailoring recommendations to the weather conditions.
\end{itemize}


\section{Requirements}
\subsection{API Choice}
\begin{itemize}
  \item Browse through the \href{https://github.com/public-api-lists/public-api-lists}{provided list}
   and choose an API of interest. This choice should be guided by the potential to retrieve, 
   manipulate, and present data in a meaningful and interactive way. I recommend choosing an 
   API that does not require authentication and is CORS enabled.\href{https://medium.com/@electra_chong/what-is-cors-what-is-it-used-for-308cafa4df1a}{(What is CORS?)}
\end{itemize}

\subsection{Project Planning}
\begin{itemize}
  \item Think through your project, researching the API documentation, project features, what data you will store, and how it will be used in your web application.
\end{itemize}

\subsection{Project Setup}
\begin{itemize}
  \item Set up a new Node.js project using Express.js.
  \item Include Axios for making HTTP requests.
  \item Include EJS for templating.
  \item Ensure that the project has a structured directory and file organization.
  \item Include pg for working with your localhost PostgreSQL database
\end{itemize}

\subsection{API Integration}
\begin{itemize}
  \item Implement at least a GET endpoint to interact with your chosen API.
  \item Use Axios to send HTTP requests to the API and handle responses.
\end{itemize}

\subsection{Database integration}
\begin{itemize}
  \item Ensure users of your web application can create accounts to access certain information
  \item Create a table relevant to the topic of your website, and include a tab in the navigation bar of 
        your web application to display this table. Your web application should also provide a means for adding data to this table.
\end{itemize}

\subsection{Data Presentation}
\begin{itemize}
  \item Design the application to present the retrieved data in a user-friendly way.
   Use appropriate HTML, CSS(bootstrap if you want), and the templating engine EJS.
\end{itemize}

\subsection{Error Handling}
\begin{itemize}
  \item Ensure that error handling is in place for both your application and any API 
  requests. You can console log any errors, but you can also give users any user-relevant errors.
\end{itemize}

\subsection{Documentation}
\begin{itemize}
  \item Include comments throughout your code to explain your logic.
  \item Include a Readme.md file that explains how to start your server, 
  what commands are needed to run your code. e.g.\textbf{ npm i  and then nodemon index.js}
\end{itemize}

\end{document}
