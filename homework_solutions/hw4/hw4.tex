\documentclass[a4paper]{article}

\usepackage{fullpage} % Package to use full page
\usepackage{parskip} % Package to tweak paragraph skipping
\usepackage{tikz} % Package for drawing
\usepackage{amsmath}
\usepackage{hyperref}

\title{CSCI 39548-03(4864) Web Development}
\author{Assignment \#4}
\date{Due: 10/2 @ 5pm}

\begin{document}
\maketitle

\section{Looping a triangle}
Write a loop that makes n calls to console.log to output the following triangle:\\
Example with n equals 7. Your program should work by only having to change the variable \textbf{n}.\\
\# \\
\#\#\\
\#\#\#\\
\#\#\#\#\\
\#\#\#\#\#\\
\#\#\#\#\#\#\\
\#\#\#\#\#\#\#\\\\
It may be useful to know that you can find the length of a string by writing .length after it.

let abc = "abc";\\
console.log(abc.length);\\
// → 3


\section{FizzBuzz}
Write a program that uses console.log to print all the numbers from 1 to 100, with two exceptions. For numbers divisible by 3, print "Fizz" instead of the number, and for numbers divisible by 5 (and not 3), print "Buzz" instead.

When you have that working, modify your program to print "FizzBuzz" for numbers that are divisible by both 3 and 5 (and still print "Fizz" or "Buzz" for numbers divisible by only one of those).

(This is actually an interview question that has been claimed to weed out a significant percentage of programmer candidates. So if you solved it, your labor market value just went up.)
\newpage

\section{Chessboard}
Write a program that creates a string that represents an 8$\times$8 grid, using newline characters to separate lines. At each position of the grid there is either a space or a "\#" character. The characters should form a chessboard.

Passing this string to console.log should show something like this:

\phantom{  }\# \# \# \#\\
\# \# \# \# \\
\phantom{  }\# \# \# \#\\
\# \# \# \# \\
\phantom{  }\# \# \# \#\\
\# \# \# \# \\
\phantom{  }\# \# \# \#\\
\# \# \# \#\\\\
When you have a program that generates this pattern, define a binding size = 8 and change the program so that it works for any size, outputting a grid of the given width and height.
\end{document}